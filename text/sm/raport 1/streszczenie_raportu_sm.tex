\documentclass[12pt]{article}
\usepackage{geometry}
\usepackage[utf8]{inputenc}
\usepackage[polish]{babel}
\usepackage[T1]{fontenc}
\usepackage{indentfirst}
\usepackage{polski}
\usepackage{graphicx} 



usepackage{polski}
usepackage{graphicx}
usepackage{graphics}
usepackage{bbm}
usepackage[utf8]{inputenc}
usepackage[a4paper, left=2.5cm, right=2.5cm, top=2.5cm, bottom=3.5cm, headsep=1.2cm]{geometry} 
usepackage{setspace}
usepackage{amsthm}
usepackage{amsmath}

\begin{document}

\title{Plantie™}
\author{Kamil Choiński, Oskar Stabla}
\maketitle

\section{Opis projektu}
Nasza koncepcja opiera się na systemie zdalnego zarządzania rośliną. Chcemy mierzyć parametry
gleby i otoczenia takie jak wilgotność, nasłonecznienie. W zależności od odczytanych wartości przez
płytkę rozwojową UNO połączoną z modułem ESP8266 będzie możliwe sterowanie pompką wody,
lampą. Mamy zamiar połączyć projekt z IT, dlatego panel sterowania będzie umieszczony na stronie
internetowej.

\section{Cel}
Nasz projekt ma na celu pomoc zabieganym ludziom, którzy nie mają czasu na zajmowanie się
swoją ukochaną roślinką przez swój częsty brak pobytu w domu. Wystarczy dostęp do internetu,
nic więcej.

\section{Kosztorys}
Suma: 58 PLN

Płytka rozwojowa UNO - 10 PLN

Moduł ESP8266 - 18 PLN

Czujnik wilgotności powietrza 5 PLN

Czujnik wilgotności gleby 5 PLN

Czujnik nasłonecznienia 5 PLN

Przetwornik 5 PLN

Żarówka 5 PLN

Kabelki, płytka uniwersalna, cyna, klej na gorąco - 10 PLN



\section{Raport prostępów prac na dzień 25.11}
Podczas ostatnich tygodni skupiliśmy się na realizacji stabilnego połączenia między płytką rozwojową UNO i modułem ESP8266, a także między modułem ESP8266 i panelem sterowania na serwerze.

Zaimplementowaliśmy pomiary wilgotności i temperatury powietrza z użyciem czujnika DHT11. Musieliśmy użyć do implementacji w ArduinoIDE biblioteki DHT-sensor-library, aby pomiary wykonywały się poprawnie.

Płytka rozwojowa UNO cały czas czeka na przychodzące dane które będą zawarte w wskaźnikach początku ''<'' i końca ''>'' danej komendy, aby wykonać określoną czynność lub odesłać pomiary wykonane przez czujnik wilgotności i temperatury do ESP8266, które jest połączone z UNO poprzez piny RX i TX. UNO odsyła swoje komendy również w wskaźnikach początku i końca, aby zapobiec ''gubieniu'' danych.

ESP8266 jest naszym tzw. ''mostem'' i w każdym momencie czeka na dane czy to z UNO czy ze strony internetowej i w zależności od kogo dane otrzymuje przerzuca je do danego odbiorcy.
Program na ESP jest zrealizowany z użyciem bibliotek ESP8266WiFi, WebSocketClient, SoftwareSerial, które kolejno służą do: umożliwieniu połączenia się ESP do sieci WiFi zadeklarowanej w naszym kodzie, połączeniu się poprzez websocket jako klient do naszego serwera postawionego na laptopie, połączenia się poprzez piny RX i TX do płytki rozwojowej UNO.

Zaimplementowaliśmy testowy kod, aby pokazać postępy naszej dotychczasowej pracy. Po naciśnięciu przycisku Water na stronie przesyłana jest komenda do UNO, które zapala diodę LED. Po naciśniuęciu przycisku Fertilise gasi ją i wysyła obecne pomiary czujnika DHT11 do serwera poprzez websocket.



\section{Wstępny harmonogram}
\subsection{4.11}
Przeanalizowanie schematów płytki rozwojowej UNO, modułu komunikacyjnego ESP8266, czujnika wilgotności gleby, czujnika wilgotności powietrza, czujnika nasłonecznienia oraz rozwiązanie techniczne doświetlania rośliny. Stworzenie schematu elektrycznego gotowego projektu. Testowanie poprawności działania posiadanych czujników w warunkach domowych. Implementacja komunikacji z modułem ESP8266. Dopasowywanie czasów działania. 
Dokupienie brakujących komponentów gotowego projektu. 
\subsection{25.11}
Realizacja połączeń elektrycznych na płytce stykowej i ostateczne testowanie poprawności dzia-
łania. Przeniesienie projektu na płytkę uniwersalną. Przygotowanie ewentualnej obudowy i realiza-
cja montażu systemu doświetlania. Stworzenie prezentacji na zajęcia

\subsection{Plany na 16.12}
Wstępna prezentacja gotowego projektu i ocena błędów.

\subsection{Plany na 20.01}
Ostateczna prezentacja z poprawką błędów.

\section{Zgodność z harmonogramem}
\subsection{4.11}
Rozpoczęliśmy prace nad naszym projektem. Postanowiliśmy skorzystać z aplikacji internetowej Trello do utworzenia tabel, które pozwoliły by nam na lepsze zarządzanie podziałem prac nad projektem.

Przeanalizowaliśmy schematy płytki rozwojowej UNO, modułu komunikacyjnego ESP8266, czujnika wilgotności gleby, czujnika wilgotności powietrza oraz czujnika nasłonecznienia. Przeanalizowaliśmy działanie każdego z czujników poprzez wykonanie przykładowych kodów na płytce rozwojowej UNO. Dowiedzieliśmy się jakich komponentów projektu nam brakuje i zamówiliśmy je w wybranych sklepach.
Stworzyliśmy wstępny schemat blokowy projektu, aby ukazać działanie poszczególnych elementów zawartych w naszym projekcie. Na ten moment postanowiliśmy nie tworzyć schematu elektrycznego, gdyż nie wiedzieliśmy jak dokładnie zrealizujemy poszczególne połączenia.

Ze względu na obszerność testowania i planowanie realizacji planu działania na nadchodzące tygodnie nie zajęliśmy się komunikacją z modułem ESP8266, ani problemem związanym z doświetlaniem naszej rośliny.


\subsection{25.11}
Do tego dnia zajmowaliśmy się komunikacją między płytką rozwojową UNO, ESP8266, a stroną internetową. Napotkaliśmy wiele problemów co nieplanowanie wydłużyło nam pracę na tym etapie projektu. Udało nam się uzyskać mniej więcej stabilne połączenie między naszymi urządzeniami. Zrealizowaliśmy połączenia UNO, ESP8266, a także podłączyliśmy do układu czujnik wilgotności i temperatury powietrza i na płytce stykowej po czym przetestowaliśmy poprawność działania. 

Przeniesienie projektu na płytkę uniwersalną okazało się niepraktyczne na tym etapie projektu, gdyż nie próbowaliśmy jeszcze połączyć ze sobą wszystkich modułów projektu i finalny kod do wgrania na UNO i ESP8266 nie był gotowy. Z powodu niewiedzy jak dokładnie będzie wyglądało nasze urządzenie na płytce uniwersalnej nie przygotowaliśmy planowanej obudowy. Nie zrealizowaliśmy również systemu doświetlania.

\subsection{16.12}
Wstępna prezentacja gotowego projektu i ocena błędów.

\subsection{20.01}
Ostateczna prezentacja z poprawką błędów.



\end{document}
